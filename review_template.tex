\documentclass[12pt]{article}
\usepackage{hyperref}
\usepackage{graphicx}
\usepackage{geometry}
    \geometry{a4paper,
    left=30mm,
    right=30mm,
    top=30mm}

\title{Review: fill in paper title}
\author{fill in names of group members}
\date{fill in date}

\begin{document}

\maketitle

\section{First Impressions}
Please rate the following aspects on a scale of 1 to 5 (1=min, 5=max).

\subsection*{Rate the overall quality of the writing:}
Rating: 4/5


\subsection*{Rate the structure of the paper:}
Rating: 4/5

\subsection*{Rate the technical quality:}
Rating: 3/5

\subsection*{Rate the general level of interest:}
Rating: 3/5


\section{Detailed Review}
\subsection*{Please provide a short summary of the paper.}
The paper covers the topic of Traffic Prediction using Graph Neural Networks. The authors start with a brief introduction to the topic 
of traffic prediction, it's importance, and the challenges that come with it. They first mention previous approaches towards the problem,
especially CNN based approaches. They then introduce the concept of Graph Neural Networks and how GNNs can be used to 
predict traffic. This is followed up by a detailed list of current challenges in the field, and the authors finish by proposing areas of future
research.

\subsection*{Please note three (or more) strong points about the paper.}
\begin{itemize}
    \item The paper is well structured and easy to follow. The sections on Challenges and Future Research are especially well written.
    \item The section on Existing approaches is very detailed and gives a good understanding of the technical aspects of the models.
    \item 
\end{itemize}


\subsection*{Please note three (or more) weak points about the paper.}
 \begin{itemize}
    \item Not enough detail on the technical aspects of the problem.
    \item More exploration needed on previous work, and references, and metrics.
    \item Diagrams and tables could be used to better explain the content.
 \end{itemize}


\subsection*{Please provide detailed comments about the weak points and concrete suggestions on how to improve them.}
\subsubsection*{Not enough detail on the technical aspects of the problem.}
The introduction section of the paper does not go into much detail about the technical aspects of the problem. The authors could also include
more references to previous work, that would help the reader in understanding the background of the problem better. Concrete data about how 
this can improve traffic management might also be helpful. 

In the beginning of Existing approaches section, it would be helpful to include a brief outline of the section, so that the reader knows what to expect.

The conclusion could also be a bit more detailed, as it is one of the first things that a reader might read, while evaluating the importance
of a paper in their own research.

\subsubsection*{More exploration needed on previous work, and references.}
The authors could include more references to previous work, that would help the reader in understanding the background of the problem better.
Some metrics that can measure the effectiveness of models could also be included, for comparative analysis. The authors could also include 
information on the standard datasets used in the field if applicable. If there are real-world implementations of the models, that could also be
mentioned.

\subsubsection*{Diagrams and Tables}
A couple of well-placed diagrams and tables could help the reader in understanding the content better.

\subsection*{Other comments (if applicable).}
There are a couple of minor spelling/grammatical mistakes.
\begin{itemize}
    \item 'However A CNN-Based ...' (Line 42) - 'A' should be lowercase.
    \item 'contet' (Line 112) 
    \item 'researches' (Line 331) - did you mean 'researchers'?
    \item 'quantificaiton' (Line 341)
    \item 'Aditionaly' (Line 350)
\end{itemize}

Additionally, the abbreviation DTW could be expanded to `Dynamic Time Warping' in the first instance of it's use, for clarity.

\subsection*{Shortly summarize your review here.}
Fill in summary here.

\end{document}

\documentclass[12pt]{article}
\usepackage{hyperref}
\usepackage{graphicx}
\usepackage{geometry}
    \geometry{a4paper,
    left=30mm,
    right=30mm,
    top=30mm}

\title{Review: Graph Neural Networks in traffic prediction problems}
\author{Ajay Narayanan, Tim Bruckdorfer, Constantin Weberpals}
\date{14.01.2024}

\begin{document}

\maketitle

\section{First Impressions}
Please rate the following aspects on a scale of 1 to 5 (1=min, 5=max).

\subsection*{Rate the overall quality of the writing:}
Rating: 4/5


\subsection*{Rate the structure of the paper:}
Rating: 4/5

\subsection*{Rate the technical quality:}
Rating: 3/5

\subsection*{Rate the general level of interest:}
Rating: 4/5


\section{Detailed Review}
\subsection*{Please provide a short summary of the paper.}
The paper covers the topic of Traffic Prediction using Graph Neural Networks. The authors start with a brief introduction to the topic 
of traffic prediction, its importance, and the challenges that come with it. They first mention previous approaches towards the problem,
especially CNN based approaches. They then introduce the concept of Graph Neural Networks and how GNNs can be used to 
predict traffic. This is followed up by a detailed list of current challenges in the field, and the authors finish by proposing areas of future
research.

\subsection*{Please note three (or more) strong points about the paper.}
\begin{itemize}
    \item The paper is well structured and easy to follow. The sections on Challenges and Future Research are especially well written and help the reader in getting a clear overview of current limitations and future research opportunities.
    \item The section on existing approaches is very detailed and gives a good understanding of the technical aspects of the models. The authors have included a profound analysis of various relevant GNN approaches, such as graph convolutional networks and spatial-temporal fusion GNNs. This adds to achieving a broad perspective on the field.
    \item The paper effectively identifies the gap in current research, particularly in the practical application of GNNs for traffic prediction. The reader is aware of the current status of research and comprehends in which direction research is heading. 
\end{itemize}


\subsection*{Please note three (or more) weak points about the paper.}
 \begin{itemize}
    \item Deeper insights on the technical aspects of the problem would be helpful.
    \item More exploration needed on previous work to provide a wide range of references.
    \item Paper might benefit from a quantitative approach, including performance metrics or direct comparisons of different models.
    \item Diagrams and tables could be used to better explain the content.
    \item Comparative analysis with non-GNN approaches is lacking and crucial for understanding why GNNs might be a good solution in this field.
    \item Inclusion of practical examples or case studies to enhance practical relevance of paper.
 \end{itemize}


\subsection*{Please provide detailed comments about the weak points and concrete suggestions on how to improve them.}
\subsubsection*{Not enough detail on the technical aspects of the problem.}
The introduction section of the paper does not go into much detail about the technical aspects of the problem. The authors could also include
more references to previous work, that would help the reader in understanding the background of the problem better. Concrete data about how 
this can improve traffic management might also be helpful. 

In the beginning of Existing approaches section, it would be helpful to include a brief outline of the section, so that the reader knows what to expect.

The conclusion could also be a bit more detailed, as it is one of the first things that a reader might read, while evaluating the importance
of a paper in their own research.

\subsubsection*{More exploration needed on previous work, references, and metrics.}
The authors could include more references to previous work, that would help the reader in understanding the background of the problem better.
Some metrics that can measure the effectiveness of models could also be included, for comparative analysis. The authors could also include 
information on the standard datasets used in the field if applicable. If there are real-world implementations of the models, that could also be
mentioned.

\subsubsection*{Diagrams and Tables}
The paper would benefit significantly from the inclusion of diagrams, flowcharts, or graphs to illustrate the complex concepts and architectures of 
GNN models. This would aid in reader comprehension, especially for those less familiar with the technical aspects. 


\subsubsection*{Comparative Analysis with Non-GNN Approaches.}
Including a comparative analysis with non-GNN approaches would provide a clearer understanding of the advantages and limitations of GNNs 
in traffic prediction. This could involve a comparison of performance metrics, computational efficiency, or scalability between GNN and traditional 
traffic prediction models. It would also help to lay a better foundation to explore potential future work, as the authors already mention that combining
GNNs with other approaches could be a potential area of research.

\subsubsection*{Inclusion of Practical Examples or Case Studies.}
Adding case studies or specific examples of GNNs being applied in real-world traffic prediction scenarios would enhance the paper's practical relevance 
and applicability. It would also provide tangible evidence of the effectiveness of GNNs in this field.

\subsection*{Other comments.}
\subsubsection*{Minor spelling/grammatical issues.}
\begin{itemize}
    \item 'Graph Neural Networks in traffic prediction problems' (Line 1) - the title should be capitalized correctly.
    \item 'However A CNN-Based ...' (Line 42) - 'A' should be lowercase.
    \item 'contet' (Line 112) 
    \item 'researches' (Line 331, 361) - did you mean 'researchers'?
    \item 'quantificaiton' (Line 341)
    \item 'Aditionaly' (Line 350)
\end{itemize}

\subsubsection*{Further improvement suggestions.}
\begin{itemize}
    \item Incorporate insights from urban planning to contextualize GNN applications within real-world traffic management scenarios.
    \item Elaborate on the types and sources of data used in GNNs for traffic prediction, addressing challenges related to data quality, availability.
    \item Clarify technical jargon for a broader audience.
    \item Highlight the novel contributions of the paper more prominently in the conclusion.
\end{itemize}

\subsection*{Shortly summarize your review here.}
This paper provides a comprehensive and well-structured overview of the use of Graph Neural Networks in traffic prediction. 
Its strengths lie in its detailed examination of existing GNN approaches and its identification of research gaps. However, the paper could be significantly improved by including comparative analyses with non-GNN methods, incorporating more visual aids, and providing practical examples or case studies. Addressing these areas would enhance the paper's depth, clarity, and practical applicability in the field of traffic prediction.

\end{document}
